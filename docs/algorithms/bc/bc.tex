\section{\textit{Betweenness Centrality}}

O algoritmo de \textit{Betweeness Centrality} é muito utilizado para o estudo 
de 
redes sociais devido a indicar a medida de centralidade para cada vértice da 
rede a que foi aplicado,isto é, calcula para cada vértice um grau de 
importância/influência. 

Para se calcular a centralidade de cada vértice é necessário calcular os 
caminhos mais curtos de todos para todos. A formula para calcular a 
\textit{Betweeness Centrality} para um dado vértice v é a seguinte:
\begin{center}
	\begin{equation}
		Bc(v) = \sum\limits_{s \neq v \neq t} 
\frac{\sigma_{st}(v)}{\sigma_{st}}
		\label{eq:bc}
	\end{equation}
	$\sigma_{st}-$~caminhos mais curtos do vértice s para o vértice t.\\
	$\sigma_{st}(v)-$ $\sigma_{st}$ dos quais passam em v.\\
\end{center}