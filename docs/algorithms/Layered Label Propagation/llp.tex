\documentclass[a4paper,10pt]{report}
\usepackage[utf8]{inputenc}


\begin{document}

\section*{\textit{Layered Label Propagation}}

  A principal ideia dos algoritmos de \textit{Label Propagation} seguem um padrão comum. Estes algoritmos consistem num conjunto de iterações e no início é atribuído a cada vértice uma \textit{label} que representa o \textit{cluster} a que pertence. No início do algoritmo, cada vértice tem uma \textit{label} diferente. O critério de atribuição da \textit{label},a cada vértice, é o que diferencia os vários algoritmos de \textit{Label Propagation}. 
  
  Um dos algoritmos mais conhecidos é o \textit{Standard Label Propagation}, em que a regra de atribuição da \textit{label} a um vértice é a \textit{label} que ocorrer mais frequentemente na sua vizinhança. Uma outra variante, denominada \textit{Absolute Pott Model}, indica que a \textit{label} que é atribuída ao vértice é a que maximiza a seguinte equação: $ki-\gamma(vi-ki)$\\
  Sendo, $ki$ os vértices na vizinhança que têm a $label_i$ e $v_i$ os todos os vértices que têm a $label_i$.
  
  Ambos os algoritmos apresentados anteriormente têm alguns problemas. O \textit{Standard Label Propagation} tende a produzir um \textit{clusters} de grandes dimensões(contendo a maior parte dos vértices) e o \textit{Absolute Pott Model} tem o problema de não se saber à partida o valor ideal para $\gamma$.
  
\end{document}
