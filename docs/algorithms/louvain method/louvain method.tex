\documentclass[a4paper,10pt]{report}
\usepackage[utf8]{inputenc}
\usepackage{mathtools}

\begin{document}

\section*{Modularidade}

Modularidade de uma partição para grafos com pesos nas arestas:

\[Q = \frac{1}{2m} \sum_{i,j} [ A_{ij} - \frac{k_i k_j}{2m} ] \delta(c_i ,c_j)\]

\begin{itemize}
	\item $A_{ij}$ é o peso da aresta entre $i$ e $j$.
	\item $k_i = \sum_j A_{ij}$ é a soma do peso de todas as arestas de $i$, com o mesmo significado para $k_j$.
	\item $c_i$ é a comunidade de $i$, com o mesmo significado para $c_j$.
	\item $\delta(c_i,c_j)$ é a igualdade entre $c_i$ e $c_j$ ( as duas vértices serem da mesma comunidade).
	\item $m = \frac{1}{2}\sum_{ij} A_{ij}$ é a soma do peso de todos as arestas %metade porque consideram que 
%o peso das arestas será calculado duas vezes, por cada vértice. Logo é necessário dividir.
\end{itemize}

A modularidade mede a densidade das arestas dentro das comunidades em comparação com aquelas entre comunidades e é usada para saber a qualidade da partição de um grafo em comunidades criadas por um método. 


\section*{Louvain Method}
Algoritmos existentes para cálculo de modularidade em grafos são muito pouco eficientes para grafos de larga escala e é também necessário detetar as comunidades dentro de comunidades (sub-comunidades).

As suas duas fases são:

\paragraph{Primeira}
Todos os vértices têm uma comunidade diferente, sendo que existem N comunidades onde N é o número de vértices.
Para cada nó é visto os seus adjacentes e calculado o ganho de modularidade que existiria se o nó fosse adicionado à comunidade do vizinho. Sendo adicionado à comunidade do vizinho com o maior ganho não negativo. Esta fase continua até já não existir ganhos (modularidade máxima).
O ganho obtido movendo um nó isolado $i$ para uma comunidade $C$ é calculado através da equação:


\[ \Delta Q  =  [\frac{\sum_{in} + k_{i,in}}{2m} - (\frac{\sum_{tot} +k_i}{2m})^2] - [\frac{\sum_{in}}{2m} - (\frac{\sum_{tot}}{2m})^2 - (\frac{k_i}{2m})^2] \]

\begin{itemize}
	\item $\sum_{in}$ é a soma do peso de todas as arestas dentro de $C$ (vão de um vértice em C para outro)
	\item $\sum_{tot}$ é a soma do peso de todas as arestas dos vértices de $C$. %( inclui Ein ou é aqueles que vão de outros vértices para os da comunidade C?)
	\item $K_i$ é a soma do peso de todas as arestas de $i$. %(incidentes...)
	\item $K_{i,in}$ é a soma do peso de todas as arestas de $i$ para vértices de $C$.
	\item $m$ é soma de do peso total do grafo.
\end{itemize}

\paragraph{Segunda}
É criado um novo grafo cujo vértices sejam as comunidades anteriormente encontradas e o peso da aresta entre duas comunidades é calculado somando o peso de todas as arestas entre estas. É depois aplicada novamente a 1ª fase a este novo grafo.
Estas duas fases são nomeadas 'passos' e continuam até ser encontrada a modularidade máxima.
\end{document}