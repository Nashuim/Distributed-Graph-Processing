\newpage
\subsection{Input/Output}
No \textit{input} é onde é definido os vários leitores para os vários formatos que os ficheiros podem ter e onde são criados os vértices e as suas arestas. Os leitores, dependente do formato, podem também atribuir aos vértices um valor e às arestas um peso.
O \textit{output} é quem define	 o formato que terá o resultado final.
\subsubsection*{Apache Giraph}
%TODO - Mudar tipos do giraph!!!!!!
O Giraph baseia-se no modelo de I/O do Apache Hadoop estando construído por cima deste mesmo modelo.
A API que é disponibilizada define para os vértices e arestas os \texttt{InputFormat}, \texttt{InputReader}, \texttt{OutputFormat} e \texttt{OutputWriter}.
Os \texttt{InputFormat} têm a responsabilidade de gerar as divisões lógicas, que depois serão usadas para repartir os dados pelos vários
\textit{workers}\footnote{\textit{slave nodes}}. É no \texttt{InputFormat} que é criado o \texttt{VertexReader}, que irá depois ser usado para criar os vértices.
A criação do vértices pode envolver a utilização de um \texttt{RecordReader} do Hadoop onde serão lidos os dados necessários a partir da informação dada pelos \texttt{InputSplit}.

O \textit{output} do Giraph é controlado pelos vários \texttt{OutputFormat} que definem um método para criar um \texttt{OutputWriter}. O \texttt{OutputWriter} pode usar um \texttt{RecordWriter}
para escrever os vértices para um \textit{output}.

\subsubsection*{Apache Hama}

O Hama também se baseia no modelo do Apache Hadoop mas constrói um modelo paralelo a este.
Este modelo consiste em disponibilizar uma API que consiste nos tipos \texttt{InputFormat}, \texttt{OutputFormat}, \texttt{VertexInputReader} e \texttt{VertexOutputReader}. Os \texttt{InputFormat} neste modelo criam objetos do tipo \texttt{RecordReader} 
que vão ler os dados que vão ser posteriormente mapeados em pares chave-valor. O \texttt{VertexInputReader} irá receber o par chave-valor para criar os
vértices.

O \textit{output} é gerado a partir de informação proveniente dos vértices utilizando o tipo \texttt{BSPeer} para agregar pares chave/valor que irão
ser escritos pelo \texttt{VertexOutputWriter}. O formato do output é controlado pelo \texttt{OutputFormat} que é especificado na
configuração do \textit{job}.

%O output é gerado a partir da informação proveniente dos vértices utilizando os BSPeers do HAMA que internamente usam o OutputFormat 
%definido pelo utilizador.

\subsubsection*{Comparação}

Apesar de ambos o modelos de I/O serem semelhantes existem algumas diferenças devido ao facto do Giraph utilizar as implementações 
do Hadoop enquanto que o Hama não. 

Em relação ao \textit{input} a principal diferença é que no \texttt{InputFormat} o Giraph cria um \texttt{Reader} para o \textit{input} desejado enquanto que 
o Hama cria um \texttt{RecordReader}. O Reader do Giraph cria o tipo desejado consoante o \textit{input} passado (um par chave/valor). O \texttt{RecordReader}
do Hama irá \textit{mapear} o input para um tuplo chave/valor que irá ser passado a um \texttt{VertexInputReader} que foi previamente configurado no \textit{job}. O \texttt{VertexInputReader}
no Hama tem a mesma funcionalidade de que o \texttt{InputReader} do Giraph.

Com o \textit{output} acontece o mesmo que com o \textit{input} em ambas as plataformas. No Giraph seleciona-se um \texttt{OutputFormat} para criar um \texttt{OutputWriter} de modo
a escrever o que é desejado. O Hama seleciona-se um \texttt{OutputFormat} que cria um \texttt{RecordWriter} de modo a obter um tuplo chave/valor que irá ser
passado a um \texttt{OutputWriter} para este gerar o \textit{output}.