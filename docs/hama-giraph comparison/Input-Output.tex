\newpage
\section*{Input/Output}

\subsection*{Apache Giraph}
%TODO - Mudar tipos do giraph!!!!!!
O Giraph baseia-se no modelo de I/O do Apache Hadoop estando construído por cima deste mesmo modelo.
A API que é disponibilizada define para os vértices e arestas os InputFormat, InputReader, OutputFormat e OutputWriter.
Os InputFormat têm a responsabilidade de gerar as divisões lógicas, que depois serão usadas para repartir os dados pelos vários
\textit{workers}\footnote{\textit{slave nodes}}. É no InputFormat que é criado o VertexReader, que irá depois ser usado para criar os vértices
A criação do vértices pode envolver a utilização de um RecordReader do Hadoop onde serão lido os dados necessários a partir da informação dada pelos
InputSplit.

O output do Giraph é controlado pelos vários OutputFormat que definem um método para criar um OutputWriter. O OutputWriter pode usar um RecordWriter
para escrever os vértices para um output.

\subsection*{Apache Hama}

O Hama também se baseia no modelo do Apache Hadoop mas constrói um modelo paralelo a este.
Este modelo consiste em InputFormat, OutputFormat, VertexInputReader, VertexOutputReader. Os InputFormat neste modelo criam RecordReader 
que vão ler os dados que vão ser posteriormente mapeados em pares chave-valor. O VertexInputReader irá receber o pare chave-valor para criar os
vértices.

O output é gerado a partir de informação proveniente dos vértices utilizando o BSPeer para agregar pares chave/valor que iram
ser escritos pelo VertexOutputWriter. O formato do output é controlado pelo OutputFormat que é especificado na
configuração do \textit{job}.

%O output é gerado a partir da informação proveniente dos vértices utilizando os BSPeers do HAMA que internamente usam o OutputFormat 
%definido pelo utilizador.

\subsection*{Comparação}