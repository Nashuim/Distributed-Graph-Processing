\newpage
\section{Input/Output}
No \textit{input} é onde é definido os vários leitores para os vários formatos que os ficheiros podem ter e onde são criados os vértices e as suas arestas. Os leitores, dependente do formato, podem também atribuir aos vértices um valor e às arestas um peso.
O \textit{output} é quem define	 o formato que terá o resultado final.
\subsection*{Apache Giraph}
%TODO - Mudar tipos do giraph!!!!!!
O Giraph baseia-se no modelo de I/O do Apache Hadoop estando construído por cima deste mesmo modelo.
A API que é disponibilizada define para os vértices e arestas os InputFormat, InputReader, OutputFormat e OutputWriter.
Os InputFormat têm a responsabilidade de gerar as divisões lógicas, que depois serão usadas para repartir os dados pelos vários
\textit{workers}\footnote{\textit{slave nodes}}. É no InputFormat que é criado o VertexReader, que irá depois ser usado para criar os vértices
A criação do vértices pode envolver a utilização de um RecordReader do Hadoop onde serão lido os dados necessários a partir da informação dada pelos
InputSplit.

O output do Giraph é controlado pelos vários OutputFormat que definem um método para criar um OutputWriter. O OutputWriter pode usar um RecordWriter
para escrever os vértices para um output.

\subsection*{Apache Hama}

O Hama também se baseia no modelo do Apache Hadoop mas constrói um modelo paralelo a este.
Este modelo consiste em InputFormat, OutputFormat, VertexInputReader, VertexOutputReader. Os InputFormat neste modelo criam RecordReader 
que vão ler os dados que vão ser posteriormente mapeados em pares chave-valor. O VertexInputReader irá receber o pare chave-valor para criar os
vértices.

O output é gerado a partir de informação proveniente dos vértices utilizando o BSPeer para agregar pares chave/valor que iram
ser escritos pelo VertexOutputWriter. O formato do output é controlado pelo OutputFormat que é especificado na
configuração do \textit{job}.

%O output é gerado a partir da informação proveniente dos vértices utilizando os BSPeers do HAMA que internamente usam o OutputFormat 
%definido pelo utilizador.

\subsection*{Comparação}

Apesar de ambos o modelos de I/O serem semelhantes existem algumas diferenças devido ao facto do Giraph utilizar as implementações 
do Hadoop enquanto que o Hama não. 

Em relação ao input a principal diferença é que no InputFormat o Giraph cria um Reader para o input desejado enquanto que 
o Hama cria um RecordReader. O Reader do Giraph cria o tipo desejado consoante o input passado (um par chave/valor). O RecordReader
do Hama irá mapear o input para um tuplo chave/valor que irá ser passado a um InputReader que foi previamente configurado no job. O InputReader
no Hama tem a mesma funcionalidade de que o InputReader do Giraph.

Com o output acontece o mesmo que com o input em ambas as plataformas. No Giraph seleciona-se um OutputFormat para criar um OutputWriter de modo
a escrever o que é desejado. O Hama seleciona-se um OutputFormat que cria um RecordWriter de modo a obter um tuplo chave/valor que irá ser
passado a um OutputWriter para este gerar o output.