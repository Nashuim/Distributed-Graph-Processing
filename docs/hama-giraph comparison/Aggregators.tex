\newpage

\section{Agregadores}
  Por vezes os algoritmos aplicados sobre redes necessitam de comunicar um estado de um \textit{superstep} para o seguinte, daí a necessidade de existir agregadores. Os agregadores são normalmente uma maneira de se obter um estado global, podendo esse estado ser um resultado computacionalmente calculado dependente dos valores agregados.
  %TODO - acabar intro?! 
  \subsection*{Apache Giraph}
    No Giraph os agregadores para puderem ser usados têm de ser registados no \texttt{MasterCompute} usando o método \texttt{registerAggregator}
    passando o nome do agregador e a classe do tipo que se quer registar. Pode-se aceder aos aggregators durante vários estágios da computação.
    Os valores dos aggregators podem ser obtidos em MasterCompute, Vertex e no WorkerContext usando o método getAggregatedValue passando
    o nome do agreggator.
    Durante os vários estágios\footnote{antes de um superstep, depois de um superstep, 
    antes do 1º superstep e depois do último superstep.} do WorkerContext e em Vertex.compute() é possível agregar valores chamando o método
    aggregate passando o nome do aggregator e o valor que irá ser agregado.
    
    No Giraph existe dois tipos de aggregators sendo um deles persistentes e o outro regulares. Nos aggregators regulares é chamado um método
    denominado reset que irá repor o valor inicial do agreggator para cada superstep. Nos aggregators persistentes os seus valores vão durar
    para toda a aplicação. Esta distinção é feita no registo do aggregator chamando registerAggregator 
    ou registerPersistentAggregator.
    
  \subsection*{Apache Hama}
    Para se usar aggregators no Hama basta registar no GraphJob a classe, que irá agregar os valores, usando o método
    setAggregatorClass. Durante os vários supersteps os vértices podem chamar o método aggregate passado um identificador único
    e o valor que irá ser agregado. Para aceder ao valor agregado de um agreggator é usado o método getAggregatedValue passando o seu 
    identificador único.
    
    O Hama permite implementar tipos de aggregators extendendo a classe genérica AbstractAggregator.
    Esta classe é parametrizada com o tipo de valor que se quer agregar e contem um conjunto de métodos que podem ser redefinidos.
    O métodos com mais relevância que dão para redefinir têm uma assinatura semelhante à seguinte:
    \begin{lstlisting}
    aggregate(Writable oldValue, Writable newValue)			  
    aggregate(Writable value);
    Writable getValue();
    \end{lstlisting}
    O método getValue é usado com o objectivo de obter o valor agregado. 
    O método aggregate que recebe dois Writables será chamado com o valor antes de ter sido feito o compute() em Vertex (oldValue) e com
    o valor posterior ao compute (newValue). O método aggregate que recebe um Writable é chamado com o valor do vértice após ter sido efectuado
    o compute. 

  \subsection*{Comparação}
  
  Apesar da ideia do Hama e do Giraph serem relativamente semelhante, existe algumas diferenças quanto à implementação. Ambas as plataformas
  permitem o registo de aggregators e de os afectar em vários estágios da computação. 
  
  Para que os \textit{aggregators} no Giraph tenham o mesmo comportameto que os presentes no Hama, os \textit{aggregators} têm de ser regulares porque o Hama não tem \textit{aggregators} persistentes. O hama, ao contrário do Giraph, não permite o registo de \textit{aggregators} parametrizados com um tipo diferente do tipo das mensagens.
  %O Hama simplifica a actualização do aggregator tendo uma interface mais completa do que o Giraph. Para que os aggregators no Giraph tenham o 
  %mesmo comportamento que os aggregators do Hama, em casos que é preciso valores pre-superstep e post-superstep, é necessário envolver 
  %entidades como o WorkerContext e fazer código para suportar as operações desejadas.
  
