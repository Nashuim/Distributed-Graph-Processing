\newpage
\section{Aresta}
Representa uma aresta num grafo, podendo esta ser pesada ou não.
\subsection*{Apache Giraph}
O Apache Giraph para suportas arcos tem um conjunto de interfaces como \texttt{Edge}, \texttt{MutableEdge}, \texttt{ReusableEdge}. A interface de \texttt{Edge} define as operações
elementares como obter o valor de um arco e o identificar do vértice a ela associado. A interface \texttt{MutableEdge} estende a interface \texttt{Edge} e define
um método para alterar o valor associado a um arco. \texttt{ReusableEdge} estende de \texttt{MutableEdge} e permite mudar o vértice que está associado ao arco.
A implementação dos arcos está feita nas classes \texttt{DefaultEdge} e \texttt{EdgeNoValue} (implementado \texttt{ReusableEdge}) e estas têm todos os métodos necessários ao uso de arestas.

\subsection*{Apache Hama}
O Hama para suportar o conceito de arco tem uma classe \texttt{Edge} que suporta um conjunto de operações de obter o valor associado ao arco, mudar o seu valor,
obter o vértice vizinho e poder muda-lo. 

\subsection*{Comparação}
Apesar do Giraph ter um conjunto maior de tipos para suportar as diversas funcionalidades de arestas, ambas fazem essencialmente
o mesmo.