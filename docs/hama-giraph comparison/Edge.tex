\subsection{Aresta}
Representa uma aresta num grafo, podendo esta ser pesada ou não.

\subsubsection*{Apache Giraph}
Para permitir o suporte de arcos, o Apache Giraph tem um conjunto de interfaces como \texttt{Edge}, \texttt{MutableEdge}, \texttt{ReusableEdge}. A interface de \texttt{Edge} define as operações
elementares como obter o valor de um arco e o identificar o vértice a ele associado. A interface \texttt{MutableEdge} deriva da interface \texttt{Edge} e define
um método para alterar o valor associado a um arco. \texttt{ReusableEdge} deriva de \texttt{MutableEdge} e permite mudar o vértice que está associado ao arco.
Existem algumas concretizações dos arcos, nomeadamente as classes \texttt{DefaultEdge} e \texttt{EdgeNoValue} que implementam \texttt{Edge} e \texttt{ReusableEdge}.

\subsubsection*{Apache Hama}
Para suportar o conceito de arco o Hama tem uma classe \texttt{Edge} que suporta um conjunto de operações, nomeadamente obter o valor associado ao arco, mudar o seu valor,
obter o vértice vizinho e modificar o arco. 

\subsubsection*{Comparação}
Apesar do Giraph ter um conjunto maior de tipos para suportar as diversas funcionalidades de arestas, ambas fazem essencialmente
o mesmo.