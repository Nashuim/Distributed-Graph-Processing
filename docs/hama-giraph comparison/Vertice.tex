\subsection{Vértice}
\label{ss:vert}
Vértice é a classe que representa uma único vértice de um grafo e é a classe principal de um algoritmo pois é nesta que está o método \texttt{compute}, o método usado para definir o algoritmo. Através desta classe é possível aceder e alterar o estado do grafo e do próprio vértice. Um vértice pode votar para parar(\textit{halt}) o algoritmo e caso não receba uma mensagem nos próximos \textit{superstep} mantém-se \textit{halted}. Sendo que o algoritmo acaba quando todos os vértices estiverem \textit{halted}.
\subsubsection*{Apache Giraph}


O Giraph apresenta a interface \texttt{Vertex} mas para realizar computações deve ser deve ser feita a implementação da interface \texttt{Computation} para possibilitar a alteração da \texttt{Computation} usada durante a execução do algoritmo, sendo que os vértices serão sempre os mesmos.

A interface \texttt{Computation} tem cinco parâmetros genéricos sendo estes \texttt{I} - o tipo do identificador do vértice, \texttt{V} - o tipo do valor do vértice, \texttt{E} - o tipo do valor das arestas, \texttt{M1} - o tipo das mensagens de entrada e \texttt{M2} - o tipo das mensagens de saída.\texttt{Vertex} apresenta \texttt{I}, \texttt{V} e \texttt{E} equivalentes.

A interface \texttt{Computation} tem como método principal \texttt{compute}, o método que deve ser implementado para realizar a computação. Este método recebe como parâmetro o vértice atual e uma sequência de mensagens para este.

Através do método \texttt{compute} o programador pode aceder aos métodos do \texttt{Vertex} que lhe permite saber e alterar o estado do vértice e das suas arestas. Também lhe permite aceder aos outros métodos de \texttt{Computation} que lhe permite alterar o estado do grafo, adicionando ou removendo vértices e arestas e conhecer o estado geral do contexto. Por exemplo, é possível saber qual o \textit{superstep}, qual o número total de vértices existentes no \textit{superstep} atual ou informação relacionada com os valores agregados. É também possível enviar mensagens aos vértices vizinhos ou a um vértice em particular.

Em adição, a interface \texttt{Computation} apresenta os métodos \texttt{preComputation} e \texttt{postComputation} que permite escrever código que será executado antes de depois da computação, respetivamente.

As computações mais comuns devem derivar de \texttt{AbstractComputation} que já apresenta a implementação dos métodos utilitários de \texttt{Computation} ou da sua subclasse \texttt{BasicComputation} que apresenta apenas um tipo de mensagem.

É possível o uso de diferentes tipos de \texttt{Computation} diferentes durante a computação alterando o tipo no \texttt{MasterCompute}.

\paragraph{Exemplo}
\begin{verbatim}
public class SimpleComputation extends 
BasicComputation<LongWritable, IntWritable, NullWritable, IntWritable> {
  @Override
  public void compute(Vertex<LongWritable, IntWritable, NullWritable> vertex,
	Iterable<IntWritable> messages) throws IOException {
    if (getSuperstep() == 1) {
		      IntWritable writable = vertex.getValue();
			      for (IntWritable msg : messages) {
				        writable.set(Math.min(writable.get(), msg.get()));
      }
      vertex.voteToHalt();
			      return;
		  }
		  sendMessageToAllEdges(vertex, vertex.getValue());
	}}
\end{verbatim}

Com esta computação de um vértice, cada vértice no 1º \textit{superstep} 
enviará para os seus adjacentes o seu valor e no 2º \textit{superstep} irá 
escolher o menor valor entre os adjacentes e o próprio vértice.

\subsubsection*{Apache Hama}
A API de vértices do Hama é definida pela classe \texttt{Vertex} em que o método principal é \texttt{compute} que permite o mesmo que o método \texttt{compute} do Giraph. Recebe apenas uma sequência de mensagens destinadas ao vértice pois não necessita receber \texttt{Vertex} sendo o próprio um \texttt{Vertex}.

A classe \texttt{Vertex} tem três parâmetros genéricos sendo estes \texttt{V} - o tipo do identificador do vértice, \texttt{E} - o tipo do valor das arestas e \texttt{M} - o tipo do valor do vértice e das mensagens.


Adicionalmente é também necessário implementar o método \texttt{setup(Configuration conf)} onde é possível definir configurações extras.

\paragraph{Exemplo}
\begin{verbatim}
public class SimpleVertex extends Vertex<LongWritable, NullWritable, 
IntWritable>{
  public void compute(Iterable<IntWritable> messages) throws IOException {
	    if(getSuperstepCount() == 1){
        IntWritable writable = getValue();
        for(IntWritable msg : messages){
	         writable.set(Math.min(msg.get(), writable.get()));
        }
        voteToHalt();
        return;
      }
      sendMessageToNeighbors(getValue());
}}
\end{verbatim}

Este exemplo tem o mesmo comportamento que o algoritmo descrito para a 
computação em giraph em que cada vértice irá ficar com o mínimo valor na sua 
adjacência.

\subsubsection*{Comparação}


As duas \textit{frameworks} diferem essencialmente porque no Hama a computação é feita através da classe \texttt{Vertex} e o Giraph tem uma segunda classe específica para computação. Isto permite que no Giraph possam ser usado vários tipos diferentes de \texttt{Computation} e assim mais facilmente definir as várias fases de um algorítmo.
No Giraph é possível criar vértices em que o seu tipo de valor é diferente do tipo das mensagens, podendo o tipo de mensagens de entrada e saída ser diferentes, enquanto que o Hama assume que estes serão sempre iguais. Relativamente à criação de um vértice no Giraph necessita de uma chamada ao método \texttt{initialize}, enquanto que no Hama a inicialização é feita automaticamente.