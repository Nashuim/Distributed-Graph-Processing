\newpage
\section*{Vertex}

\subsection*{Apache Giraph}


O Giraph apresenta a interface \texttt{Vertex<I extends WritableComparable, V extends Writable, E extends Writable>} mas para realizar computações deve ser deve ser feita a implementação da interface \texttt{Computation<I extends WritableComparable, V extends Writable, E extends Writable, M1 extends Writable, M2 extends Writable>}.

A interface \texttt{Computation} tem cinco parâmetros genéricos sendo estes \texttt{I} - o tipo do identificador do vértice, \texttt{V} - o tipo do valor do vértice, \texttt{E} - o tipo do valor das arestas, \texttt{M1} - o tipo das mensagens de entrada e \texttt{M2} - o tipo das mensagens de saída, sendo \texttt{I}, \texttt{V} e \texttt{E} de \texttt{Vertex} equivalentes.

A interface \texttt{Computation} tem como método principal \texttt{compute(Vertex<I, V, E> vertex, Iterable<M1> messages)}, o método que deve ser implementado para realizar a computação. Este método recebe como parâmetro o vértice actual e as mensagens para este.

Através do método \texttt{compute} o programador pode aceder aos métodos do \texttt{Vertex} que o permite saber e alterar o estado do vértice e das suas arestas. E também aceder aos outros métodos de \texttt{Computation} que o permite alterar o estado do grafo adicionando ou removendo vértices e arestas e conhecer o estado geral do contexto como o qual o \textit{superstep} atual, qual o número total de vértices existentes no \textit{superstep} corrente ou informação relacionada com valores agregados. É também possível enviar mensagens aos vértices vizinhos ou a um vértice em particular.

Em adição a interface \texttt{Computation} apresenta os métodos \texttt{preComputation()} e \texttt{postComputation()} que permite fazer código que será executado antes de depois da computação, respetivamente.

As computações mais comuns devem estender de \texttt{AbstractComputation} que já apresenta a implementação dos métodos utilitários de \texttt{Computation} ou da sua subclasse \texttt{BasicComputation} que apresenta apenas um tipo de mensagem.
Em similaridade deve-se estender de \texttt{DefaultVertex} pois este apresenta os métodos utilitários já implementados.

É possível o uso de várias classes \texttt{Computation} diferentes durante a computação alterando a classe no \texttt{MasterCompute}.

\subsection*{Apache Hama}
A API de vértices do Hama é definida pela classe \texttt{Vertex<V extends WritableComparable, E extends Writable, M extends Writable>} em que o método principal é \texttt{compute(Iterable<M> messages)} que permite o mesmo que o método \texttt{compute} do Giraph, tendo acesso directo aos métodos de \texttt{Vertex}

A classe \texttt{Vertex} tem três parâmetros genéricos sendo estes \texttt{V} - o tipo do identificador do vértice, \texttt{E} - o tipo do valor das arestas e \texttt{M} - o tipo do valor do vértice e das mensagens.


Adicionalmente é também necessário implementar o método \texttt{setup(Configuration conf)} onde é possível definir configurações extras.
\subsection*{Comparação}


As duas \textit{frameworks} diferem em que no Hama a computação é feita através da classe \texttt{Vertex} e o Giraph tem uma segunda classe específica para computação. Isto permite que no Giraph possa ser usado vários tipos diferentes de \texttt{Computation} e assim mais facilmente definir várias fases de um algorítmo.
No Giraph é possível criar vértices em que o seu tipo de valor é diferente do tipo das mensagens, podendo o tipo de mensagens de entrada e saída ser diferentes, mas o Hama assume que estes serão sempre iguais. A criação de um vértice no Giraph necessita de uma chamada ao método \texttt{initialize}, no Hama a inicialização é feita automaticamente.