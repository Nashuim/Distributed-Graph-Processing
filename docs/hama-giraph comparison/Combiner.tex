\newpage
\subsection{Combinadores}
O envio de mensagens entre vértices pode envolver o custo do envio sobre a rede ou pode existir um grande número de mensagens. Se o tipo de mensagem demonstrar uma propriedade comutativa é possível combinar várias mensagens numa única e enviar apenas esta mensagem, reduzindo o custo associado com o envio de mensagens. Os \textit{Combiners} servem para tal, combinando mensagens destinadas a um vértice.

\subsubsection*{Apache Giraph}
Esta plataforma apresenta uma interface de combinadores binários que define dois métodos \texttt{combine} e \texttt{createInitialMessage}.

O método \texttt{combine} é o método principal e recebe como parâmetro um identificador do vértice a que se destina as mensagens e duas mensagens a combinar.

O método \texttt{createInitialMessage} deve retornar uma mensagem cujo valor seja um elemento neutro da operação do combinador (ex: Um combinador para somas deve retornar 0).

Quando é especificado um combinador, as mensagens armazenadas são combinadas de modo a que o vértice receba apenas uma mensagem.

Adicionalmente, é possível mudar o tipo de combinador usado durante a computação no MasterCompute.

\paragraph{Exemplo}
\begin{verbatim}
public class MinIntCombiner extends MessageCombiner<LongWritable, IntWritable>{
	@Override
	public void combine(LongWritable vertex, IntWritable originalMessage, 
IntWritable otherMessage) {
		originalMessage.set(Math.min(originalMessage.get(), 
otherMessage.get()));
	}
	@Override
	public IntWritable createInitialMessage() {
		return new IntWritable(Integer.MAX_VALUE);
	}
}
\end{verbatim}

Com este combinador o vértice de destino apenas receberia a mensagem que 
tivesse o menor valor.

\subsubsection*{Apache Hama}
Esta plataforma apresenta uma interface de combinador de mensagens múltiplas que define o único método \texttt{combine}.

O método \texttt{combine} recebe uma sequência de mensagens destinadas a um 
vértice e deve retornar uma única mensagem como o resultado da sua operação de 
combinação.

\paragraph{Exemplo}
\begin{verbatim}
public class MinIntCombiner extends Combiner<IntWritable>{
	@Override
	public IntWritable combine(Iterable<IntWritable> messages) {
		IntWritable writable = new IntWritable(Integer.MAX_VALUE);
		for(IntWritable msg : messages){
			writable.set(Math.min(writable.get(), msg.get()));
		}	
		return writable;
	}
}
\end{verbatim}
Para este exemplo, um vértice apenas receberia a mensagem que tivesse menor 
valor.

\subsubsection*{Comparação}
As diferenças entre os combinadores Giraph e Hama resulta principalmente do tipo de combinação pedida ao programador. Sendo que no Giraph é apenas necessário combinar duas mensagens de cada vez mas no Hama é possível a combinação de várias mensagens em simultâneo.

Isto pode levar a código de combinadores mais simples no Giraph. No outro, lado a API do Giraph pede a implementação de dois métodos, em comparação com o Hama onde existe apenas um.

No Hama só é possível usar um tipo de combinador por computação mas no Giraph podem ser usados vários.

Para finalizar, o método \texttt{combine} do Giraph recebe também o \textit{id} do vértice, algo que pode ser contra-intuitivo visto que operações de combinação raramente dependem do vértice de destino.
