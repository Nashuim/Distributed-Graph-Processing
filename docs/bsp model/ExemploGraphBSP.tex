\section{Exemplo da interface programável para grafos}

Para exemplificar o uso da interface que é disponibilizada para se 
realizar algoritmos em grafos,seguindo o modelo BSP, pode-se observar o 
algoritmo \textit{Shortest Path}, implementado em Apache Giraph, que indica a 
distância de um vértice fonte para todos os outros vértices.
\newpage
\begin{verbatim}
public class ShortestPathComputation extends BasicComputation<LongWritable, 
IntWritable, NullWritable, IntWritable>{
  @Override
  public void compute(Vertex<LongWritable, IntWritable,NullWritable> vertex, 
	Iterable<IntWritable> msgs) throws IOException {
	
	  if(getSuperstep() == 0){
	    if(vertex.getId().get() == 0){
	      // O vértice fonte envia mensagem para todos os vértices 
	      // adjacentes com distância 1.
	      sendMessageToAllEdges(vertex, new IntWritable(1));
	    }else{
	      // A distância inicial do vértice fonte para todos os outros 
	      // vértices é infinita.
	      vertex.setValue(new IntWritable(Integer.MAX_VALUE));
							}
		}
	  if(getSuperstep() > 0){
	    // Apenas vértices que tenham recebido mesangens iram estar 
	    // acordados.
	    IntWritable val = vertex.getValue();		
	    for(IntWritable rcvDistance : msgs){
	      // O vértice fica com a distância mínima até ao vértice fonte.
	      val.set(Math.min(val.get(), rcvDistance.get()));
	    }
	    // Envia para os vértices adjacentes a distância mínima que tem até 
	    // ao vértice fonte.
	    sendMessageToAllEdges(vertex, val);		
		} 
		vertex.voteToHalt();
		}}
\end{verbatim}

