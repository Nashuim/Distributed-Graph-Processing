\section*{ Calendariza��o}
O planeamento do projecto � o seguinte:
\begin{table}[h]
 \caption{ Calendariza��o semanal}
 \begin{tabular}{|l|l|p{9.5cm}|}
 \hline
 \bf{Data de Inicio} & \bf{Semana} & \bf{Descri��o} \\ \hline
 2 Mar�o & 1 & Escrita da Proposta \\ \hline
 9 Mar�o & 2 & Finaliza��o da Proposta e inicia��o do estudo das Plataformas \\ \hline
 16 Mar�o & 3 & Estudo das Plataformas e levantamento das interfaces program�veis. \\ \hline
 23 Mar�o & 4 & Estudo dos algoritmos \textit{k-core}, \textit{heat kernel} e \textit{BFS}  \\ \hline
 30 Mar�o & 5-6 & Estudo dos algoritmos \textit {Layered Label Propagation} e \textit{Louvain}. Estruturar os v�rios m�dulos da biblioteca.\\ \hline
 13 Abril & 7-8 & Implementa��o dos algoritmos estudados na 4� semana.\\ \hline
 27 Abril & 9-10 & Prepara��o da apresenta��o individual e relat�rio de progresso \\ \hline
 11 Maio & 11-12 & Implementa��o dos algoritmos \textit{Louvain} e \textit{Layered Label Propagation}.\\ \hline
 25 Maio & 13-14 & Estudar e implementar o algoritmo de \textit{Betweenness Centrality}. \\ \hline
 8 Junho & 15 & Cartaz e finaliza��o da vers�o com a implementa��o dos algoritmos obrigat�rios. \\ \hline
 15 Junho & 16-19 & Escolha dos \textit{data-sets} (que iram ser usados nos testes), testes e compara��es com outras plataformas. \\ \hline
 13 Julho & 20 & Finaliza��o do relat�rio e entrega da vers�o final. \\ \hline
\end{tabular}
\end{table}
\\[0.5cm]
O relat�rio ir� ser realizado de forma gradual, havendo contribui��es em todas as semanas.