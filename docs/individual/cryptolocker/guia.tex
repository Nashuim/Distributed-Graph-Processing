\documentclass[a4paper,10pt]{report}
\usepackage[utf8]{inputenc}
\begin{document}

\begin{enumerate}
	\item Eu sou ... e vou falar sobre Cryptolocker.
	\item Cryptolocker é um ransomware que apareceu em meios de Setembro de 2013, afeta sistemas Windows,
e ao contrário da maioria dos ransomwares naquela altura que apenas bloqueavam o acesso
do utilizador ao seu computador, o Cryptolocker também encripta certos tipos de ficheiros.
	Esta encriptação é feita usando um tipo de cifra assimétrica, ou seja, não é possível
desencriptar os ficheiros sem a chave privada.

	\item Esta é a imagem que aparece aos utilizadores após serem infetados pelo Cryptolocker.
	Diz que os seus ficheiros foram encriptados e , a não ser que paguem dentro do tempo limite,
a chave de decriptação será destruída e o acesso aos ficheiros será perdido.

	\item Quando o Cryptolocker inicia a sua execução começa por adicionar entradas que o permitem 
executar no startup do sistema e também no modo de segurança.
Estas são as duas entradas usadas para tal nas versões mais recentes do malware.

	\item Cria também um registo extra para configurações. Neste registo vão ficar coisas 
como a informação sobre a versão e a chave de encriptação, após ser conseguida.
	Finalmente, cria um executável seu com um nome aleatório em AppData ou LocalAppData, dependendo da versão
do malware.

	\item Depois, começa a tentar contratar o servidor de controlo para conseguir a chave de encriptação. 
Para encontrar o servidor usa um algoritmo de geração de domínio.
	Após ter a chave o Cryptolocker começa a encriptação dos ficheiros.

	\item A encriptação em si é muito simples. Para cada ficheiro encontrado é gerado uma chave AES de 256 bits 
que é usada para encriptar o ficheiro. A chave AES é encriptada pela chave pública conseguida do servidor
e a chave AES encriptada, dados extras e o ficheiro encriptado substituem o ficheiro original.
	São encriptados todos os ficheiros encontrados em drives locais e remotas destes tipos.

	\item A maioria do tipos de ficheiros são tipos mais comuns em empresa, não tendo muitos tipos de mídia
encontrados em computadores de utilizadores comuns. Embora ataques a estes não sejam muito incomuns.

	\item Para cada ficheiro encontrado é criado mais uma entrada no registo com o caminho do ficheiro.
	O Cryptolocker continua a encriptação em escondido até encriptar encriptar todos os ficheiros 
e só ai mostra-se ao utilizador com aquela imagem mostrada anteriormente.

	\item A infeção acontece principalmente por email.O email contém como anexo um ficheiro zip que tem
um executável normalmente mascarado como PDF. O nome da extensão e a imagem do ficheiro são todos alterados para tal. 

O executável não é o Cryptolocker mas sim um outro programa que irá fazer download e executar o
Cryptolocker.

	\item No final do ano passado a maior parte das infeções aconteceram em países de língua inglesa embora a infeção nos outros países tem estado a aumentar.

	\item Para terminar, embora não sejam possíveis recuperar os ficheiros, é possível prevenir o Cryptolocker.
	
	Uma forma óbvia é backups frequentes para que se possa repor os ficheiros e outra
é impedir a execução de executáveis em AppData e LocalAppData.

(Mais, depende do tempo)
	É para mostrar que embora o que o Cryptolocker em si faz é simples, encriptar os ficheiros do
utilizador, as consequências de tais pode levar a um custo alto.
	De Setembro a Dezembro do ano passado estima-se que os atacantes conseguiram mais do que 42 milhões de 
dólares.
\end{enumerate}
\end{document}