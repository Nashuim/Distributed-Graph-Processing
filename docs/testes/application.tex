\chapter{Aplicações a grandes redes e Observações}
De modo a verificar a validade dos nossos algoritmos estes foram aplicados a várias redes de testes verificando o tempo de computação e o uso de memória (ver tabela~\ref{tbl:tm}) na plataforma Giraph.
\begin{table}
	\centering
 \begin{tabular}{|l|l|l|l|l|l|l|}
 \hline
			Algoritmo & 250k & 500k & 750k & 1000k & Twitter - 1 & Twitter - 2\\ \hline
			Louvain Distribuído&334,4s&589,2s&663s&811s&754s&1737,2s\\ \hline
			Louvain Original&6s&20,4s&41,6s&66,6s&0&0\\ \hline
			LLP Distribuído& 0 & 0 &0 &0&0&0\\ \hline
			LLP Paralelo&0&0&0&0&0&0 \\ \hline
		\end{tabular}
		\caption{Esta tabela mostra o desempenho dos nossos algoritmos, da implementação original do Louvain \textit{Method}~\cite{orgLouvain} e de uma implementação paralela na plataforma WebGraph do Layered Label Propagation~\cite{prlLLP}}
		\label{tbl:tm}
\end{table}

 As redes usadas são redes densas geradas para teste de diferente tamanhos e duas redes reais resultante de de mensagens no Twitter entre utilizadores com 1,33 milhões de nós e [FALTA TAMANHO] de nós. Em todos os casos foi observado um maior tempo de computação nas nossas versões distribuídas dos algoritmos.

Testes do Louvain Distribuído mostraram que a opção de modularidade mínima permitida varia o tempo de computação até um valor máximo onde este estabiliza, faltando testar a qualidade das comunidades geradas com diversos valores de modularidade mínima. 
Sendo que a modularidade mínima permitida modifica as mudanças de comunidade por cada nó foi também observado este pode levar a poucas ou nenhuma mudanças quando maior ou igual que um certo valor. Sendo este valor 1 para as redes densas testados mas as redes do Twitter sendo esparsas, mostram responder melhor a valores superior a 1. Valores muito pequenos levaram a diversas mudanças e por isso, um grande tempo de computação.
